\scnheader{Реализация подсистемы взаимодействия с внешней средой с использованием сетевых протоколов}
\scnrelfromlist{компонент программной системы}{РРеализация подсистемы взаимодействия с внешней средой с использованием протоколов на основе формата JSON}
\scnexplanation{Взаимодействие программной модели sc-памяти с внешними ресурсами может осуществляться посредством специализированного программного интерфейса (API), однако этот вариант неудобен в большинстве случае, поскольку:
    \begin{scnitemize}
        \item поддерживается только для очень ограниченного набора языков программирования (С, С++, Python);
        \item требует того, чтобы клиентское приложение, обращающееся к программной модели sc-памяти, фактически составляло с ней единое целое, таким образом исключается возможность построения распределенного коллектива ostis-систем;
        \item как следствие предыдущего пункта, исключается возможность параллельной работы с sc-памятью нескольких клиентских приложений.
    \end{scnitemize}

Для того, чтобы обеспечить возможность удаленного доступа к sc-памяти не учитывая при этом языки программирования, с помощью которых реализовано конкретное клиентское приложение, было принято решение о реализации возможности доступа к sc-памяти с использованием универсальных протоколов, не зависящих от средств реализации того или иного компонента или системы. В качестве такого протокола был разработан текстовый протокол на базе JSON.}

\scnheader{Реализация подсистемы взаимодействия с внешней средой с использованием протоколов на основе формата JSON}
\scnexplanation{В связи с большим числом недостатков протокола SCTP было принято решение о разработке другого протокола на основе какого-либо общепринятого текстового транспортного формата. В качестве такого формата был выбран формат JSON.}
\scnrelto{реализация}{Протокол взаимодействия с sc-памятью на основе JSON}
\scnaddlevel{1}
\scnnote{Данный протокол пока не имеет собственного названия}
\scntext{программная документация}{http://ostis-dev.github.io/sc-machine/http/websocket/}
\scnexplanation{В рамках \textit{Протокола взаимодействия с sc-памятью на основе JSON} каждая команда представляет собой json-объект, в котором указываются идентификатор команда, тип команды и ее аргументы. В свою очередь ответ на команду также представляет собой json-объект, в котором указываются идентификатор команды, ее статус (выполнена успешно/безуспешно) и результаты. Структура аргументов и результатов команды определяется типом команды.}
\scnrelfromlist{достоинство}{\scnfileitem{JSON является общепринятым открытым форматом, для работы с которым существует большое количество библиотек для популярных языков программирования. Это, в свою очередь, упрощает реализацию клиента и сервера для протокола, построенного на базе JSON.};
\scnfileitem{Реализация протокола на базе JSON не накладывает принципиальных ограничений на объем (длину) каждой команды, в отличие от бинарного протокола. Таким образом, появляется возможность использования неатомарных команд, позволяющих, например, за один акт пересылки такой команды по сети создать сразу несколько sc-элементов. Важными примерами таких команд являются  \textit{Команда генерации по произвольному образцу} и \textit{Команда поиска по произвольному образцу}.}}
\scnnote{Можно сказать, что протокол на базе JSON является следующим шагом на пути к созданию мощного и универсального языка запросов, аналогичного языку SQL для реляционных баз данных и предназначенному для работы с sc-памятью. Следующий шагом станет реализация такого протокола на основе одного из стандартов внешнего отображения sc-конструкций, например, \textit{SCs-кода}, что, в свою очередь, позволит передавать в качестве команд целые программы обработки sc-конструкций, например на языке SCP.}
\scnaddlevel{-1}

