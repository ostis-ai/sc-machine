\scnheader{Реализация вспомогательных инструментальных средств для работы с sc-памятью}
\scnrelfrom{компонент программной системы}{Реализация сборщика базы знаний из исходных текстов, записанных в SCs-коде}
\scnaddlevel{1}
\scnidtf{sc-builder}
\scnrelfrom{используемый язык}{SCs-код}
\scnexplanation{Сборщик базы знаний из исходных текстов позволяет осуществить сборку базы знаний из набора исходных текстов, записанных в SCs-коде с ограничениями (см. \textit{Раздел **про исходные тексты**}) в бинарный формат, воспринимаемый \textit{Программной моделью sc-памяти}. При этом возможна как сборка "с нуля"{} (с уничтожением ранее созданного слепка памяти), так и аддитивная сборка, когда информация, содержащаяся в заданном множестве файлов, добавляется к уже имеющемуся слепку состояния памяти.

В текущей реализации сборщик осуществляет "склеивание"{} ("слияние"{}) sc-элементов, имеющих на уровне исходных текстов одинаковые \textit{системные sc-идентификаторы}.}
\scnaddlevel{-1}
